\documentclass{article}
\usepackage{fancyhdr}
\usepackage{xeCJK}
\usepackage{pifont}
\usepackage{graphicx}
\usepackage{float}
\usepackage{geometry}
\geometry{left=1.5cm,right=1.5cm,top=3cm,bottom=3cm}
\usepackage[colorlinks,linkcolor=blue]{hyperref}
%\setmainfont{Times New Roman}  
\setCJKmainfont{Songti SC}
\pagestyle{fancy}
\fancypagestyle{plain}{
    \fancyhf{}
    \fancyfoot[C]{\thepage}
    \renewcommand\headrulewidth{0pt}
}
\begin{document}
    \noindent\textbf{6.12}指令中的操作数可以存放在哪些地方?\par
    \ding{172} 以立即数方式存放在指令中\par
    \ding{173} cpu内部寄存器中\par
    \ding{174} 存储器中
    \\[4pt]\par

    \noindent\textbf{6.15}对比寄存器偏移寻址、前变址寻址和后变址寻址的异同。\par
    相同:三者都基于寄存器偏移寻址,指令的格式类似\par
    不同:寄存器偏移寻址是指在执行指令时,操作数地址为基地址加偏移量,指令完成后不改变基地址和地址偏移两个寄存器的值;前变址寻址是指在执行指令时,将基地址加偏移量得到的地址写回基地址寄存器,然后按照新的基地址来寻址;后变址寻址是指在执行指令时,按照基地址先寻址,完成后再将基地址加偏移量写回基地址寄存器。
    \\[4pt]\par

    \noindent\textbf{6.20}解释汇编语法“LDM \{addr\_mode\} <Rn>\{!\}, <registers>”中各部分要素的含义。\par
    \ding{172} LDM表示多寄存器加载指令\par
    \ding{173} \{addr\_mode\}为可选后缀,表示地址模式,有IA、IB、DA、DB四种模式\par
    \ding{174} <Rn>表示选用的基地址寄存器\par
    \ding{175} \{!\}表示是否将新修改的地址写回到基地址寄存器中\par
    \ding{176} <registers>为载入数据的寄存器集合
    \\[4pt]\par

    \noindent\textbf{6.34}什么是符号扩展?\par
    进行数扩展时,扩展位和符号为保持一致,保留有符号数的符号位。
    \\[4pt]\par

    \noindent\textbf{6.35}MOV指令是否可以完成从一个存储器单元到一个寄存器的数据传送?为什么?\par
    不能。MOV只支持处理器内部的电路单元之间的数据传送,包括寄存器之间、立即数到寄存器、通用寄存器和特殊寄存器之间等。
    \\[4pt]\par

    \noindent\textbf{6.40}为什么“BL”或“BLX”指令适用于函数调用,而“B”指令不适合。\par
    B表示无条件跳转,不带其他操作,比如保存返回地址以返回调用语句下一句等,因此单个B指令不适合函数调用;而BL和BLX则会先将返回地址压入栈中,解决了返回的问题。
    \\[4pt]\par

    \noindent\textbf{6.42}“CBZ”指令的作用与“CMP”指令组合“BEQ”指令有什么区别?\par
    CBZ在跳转后不影响APSR,且只能单向向前跳转4-130字节的指令,只能使用R0-R7,适用于小范围循环控制;后两者会更新APSR,可以实现任意位置跳转,适用于大范围控制。
    \\[4pt]\par

    \noindent\textbf{6.43}解释饱和和溢出的区别。\par
    饱和:数据达到可表示的最值时不再增或者减,保持在最值。加减运算有饱和的概念。\par
    溢出:传统运算方式中,数据达到可表示最值后继续增或减,自动回到计数零点。
    \\[4pt]\par

    \noindent\textbf{Hint:}\quad View this HW github repositry at:\par
    \quad \url{https://github.com/cabasky/2021F-Embedded_System_HW}
    \\[4pt]\par
\end{document}